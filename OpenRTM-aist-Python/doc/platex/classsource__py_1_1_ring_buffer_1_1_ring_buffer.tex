\section{クラス RingBuffer}
\label{classsource__py_1_1_ring_buffer_1_1_ring_buffer}\index{source\_\-py::RingBuffer::RingBuffer@{source\_\-py::RingBuffer::RingBuffer}}
リングバッファ実装クラス  


\subsection*{Public メソッド}
\begin{CompactItemize}
\item 
def {\bf \_\-\_\-init\_\-\_\-}
\begin{CompactList}\small\item\em コンストラクタ \item\end{CompactList}\item 
def {\bf init}
\begin{CompactList}\small\item\em 初期化 \item\end{CompactList}\item 
def {\bf clear}
\begin{CompactList}\small\item\em クリア \item\end{CompactList}\item 
def {\bf length}
\begin{CompactList}\small\item\em バッファ長を取得する \item\end{CompactList}\item 
def {\bf write}
\begin{CompactList}\small\item\em バッファに書き込む \item\end{CompactList}\item 
def {\bf read}
\begin{CompactList}\small\item\em バッファから読み出す \item\end{CompactList}\item 
def {\bf isFull}
\begin{CompactList}\small\item\em バッファが満杯であるか確認する \item\end{CompactList}\item 
def {\bf isEmpty}
\begin{CompactList}\small\item\em バッファが空であるか確認する \item\end{CompactList}\item 
def {\bf isNew}
\begin{CompactList}\small\item\em 最新データか確認する \item\end{CompactList}\item 
def {\bf put}
\begin{CompactList}\small\item\em バッファにデータを格納する \item\end{CompactList}\item 
def {\bf get}
\begin{CompactList}\small\item\em バッファからデータを取得する \item\end{CompactList}\item 
def {\bf getRef}
\begin{CompactList}\small\item\em 次に書き込むバッファへの参照を取得する \item\end{CompactList}\end{CompactItemize}
\subsection*{データ構造}
\begin{CompactItemize}
\item 
class {\bf Data}
\begin{CompactList}\small\item\em バッファデータクラス \item\end{CompactList}\end{CompactItemize}


\subsection{説明}
リングバッファ実装クラス 



\footnotesize\begin{verbatim}
\end{verbatim}
\normalsize


指定した長さのリング状バッファを持つバッファ実装クラス。 バッファ全体にデータが格納された場合、以降のデータは古いデータから 順次上書きされる。 従って、バッファ内には直近のバッファ長分のデータのみ保持される。

注)現在の実装では、一番最後に格納したデータのみバッファから読み出し可能

\begin{Desc}
\item[引数:]
\begin{description}
\item[{\em DataType}]バッファに格納するデータ型\end{description}
\end{Desc}
\begin{Desc}
\item[から:]0.4.0 \end{Desc}


 RingBuffer.py の 40 行で定義されています。

\subsection{関数}
\index{source\_\-py::RingBuffer::RingBuffer@{source\_\-py::RingBuffer::RingBuffer}!\_\-\_\-init\_\-\_\-@{\_\-\_\-init\_\-\_\-}}
\index{\_\-\_\-init\_\-\_\-@{\_\-\_\-init\_\-\_\-}!source_py::RingBuffer::RingBuffer@{source\_\-py::RingBuffer::RingBuffer}}
\subsubsection{\setlength{\rightskip}{0pt plus 5cm}def \_\-\_\-init\_\-\_\- ( {\em self},  {\em length})}\label{classsource__py_1_1_ring_buffer_1_1_ring_buffer_c775ee34451fdfa742b318538164070e}


コンストラクタ 

コンストラクタ 指定されたバッファ長でバッファを初期化する。 ただし、指定された長さが2未満の場合、長さ2でバッファを初期化する。

\begin{Desc}
\item[引数:]
\begin{description}
\item[{\em self}]\item[{\em length}]バッファ長 \end{description}
\end{Desc}


 RingBuffer.py の 61 行で定義されています。\index{source\_\-py::RingBuffer::RingBuffer@{source\_\-py::RingBuffer::RingBuffer}!init@{init}}
\index{init@{init}!source_py::RingBuffer::RingBuffer@{source\_\-py::RingBuffer::RingBuffer}}
\subsubsection{\setlength{\rightskip}{0pt plus 5cm}def init ( {\em self},  {\em data})}\label{classsource__py_1_1_ring_buffer_1_1_ring_buffer_c95cf511c1f2ff09ca47fe587657c2b4}


初期化 

バッファの初期化を実行する。 指定された値をバッファ全体に格納する。

\begin{Desc}
\item[引数:]
\begin{description}
\item[{\em self}]\item[{\em data}]初期化用データ \end{description}
\end{Desc}


 RingBuffer.py の 88 行で定義されています。\index{source\_\-py::RingBuffer::RingBuffer@{source\_\-py::RingBuffer::RingBuffer}!clear@{clear}}
\index{clear@{clear}!source_py::RingBuffer::RingBuffer@{source\_\-py::RingBuffer::RingBuffer}}
\subsubsection{\setlength{\rightskip}{0pt plus 5cm}def clear ( {\em self})}\label{classsource__py_1_1_ring_buffer_1_1_ring_buffer_07b95aa63e9e2d286ef0aa83d5bb34b2}


クリア 

バッファに格納された情報をクリアする。

\begin{Desc}
\item[引数:]
\begin{description}
\item[{\em self}]\end{description}
\end{Desc}


 RingBuffer.py の 105 行で定義されています。\index{source\_\-py::RingBuffer::RingBuffer@{source\_\-py::RingBuffer::RingBuffer}!length@{length}}
\index{length@{length}!source_py::RingBuffer::RingBuffer@{source\_\-py::RingBuffer::RingBuffer}}
\subsubsection{\setlength{\rightskip}{0pt plus 5cm}def length ( {\em self})}\label{classsource__py_1_1_ring_buffer_1_1_ring_buffer_de624967a623600100475a859a3cc0a4}


バッファ長を取得する 

バッファ長を取得する。

\begin{Desc}
\item[引数:]
\begin{description}
\item[{\em self}]\end{description}
\end{Desc}
\begin{Desc}
\item[戻り値:]バッファ長 \end{Desc}


 RingBuffer.py の 125 行で定義されています。\index{source\_\-py::RingBuffer::RingBuffer@{source\_\-py::RingBuffer::RingBuffer}!write@{write}}
\index{write@{write}!source_py::RingBuffer::RingBuffer@{source\_\-py::RingBuffer::RingBuffer}}
\subsubsection{\setlength{\rightskip}{0pt plus 5cm}def write ( {\em self},  {\em value})}\label{classsource__py_1_1_ring_buffer_1_1_ring_buffer_8c7fc40f1124fcf3f5ee7116cd62f413}


バッファに書き込む 

引数で与えられたデータをバッファに書き込む。

\begin{Desc}
\item[引数:]
\begin{description}
\item[{\em self}]\item[{\em value}]書き込み対象データ\end{description}
\end{Desc}
\begin{Desc}
\item[戻り値:]データ書き込み結果(常にtrue:書き込み成功を返す) \end{Desc}


 RingBuffer.py の 146 行で定義されています。\index{source\_\-py::RingBuffer::RingBuffer@{source\_\-py::RingBuffer::RingBuffer}!read@{read}}
\index{read@{read}!source_py::RingBuffer::RingBuffer@{source\_\-py::RingBuffer::RingBuffer}}
\subsubsection{\setlength{\rightskip}{0pt plus 5cm}def read ( {\em self},  {\em value})}\label{classsource__py_1_1_ring_buffer_1_1_ring_buffer_fcea8c9091d60bee0f6fb79abb6e1cca}


バッファから読み出す 

バッファに格納されたデータを読み出す。

\begin{Desc}
\item[引数:]
\begin{description}
\item[{\em self}]\item[{\em value}]読み出したデータ\end{description}
\end{Desc}
\begin{Desc}
\item[戻り値:]データ読み出し結果 \end{Desc}


 RingBuffer.py の 168 行で定義されています。\index{source\_\-py::RingBuffer::RingBuffer@{source\_\-py::RingBuffer::RingBuffer}!isFull@{isFull}}
\index{isFull@{isFull}!source_py::RingBuffer::RingBuffer@{source\_\-py::RingBuffer::RingBuffer}}
\subsubsection{\setlength{\rightskip}{0pt plus 5cm}def isFull ( {\em self})}\label{classsource__py_1_1_ring_buffer_1_1_ring_buffer_4928b8e993df1ab082183bda426cb9b1}


バッファが満杯であるか確認する 

バッファ満杯を確認する。(常にfalseを返す。)

\begin{Desc}
\item[引数:]
\begin{description}
\item[{\em self}]\end{description}
\end{Desc}
\begin{Desc}
\item[戻り値:]満杯確認結果(常にfalse) \end{Desc}


 RingBuffer.py の 191 行で定義されています。\index{source\_\-py::RingBuffer::RingBuffer@{source\_\-py::RingBuffer::RingBuffer}!isEmpty@{isEmpty}}
\index{isEmpty@{isEmpty}!source_py::RingBuffer::RingBuffer@{source\_\-py::RingBuffer::RingBuffer}}
\subsubsection{\setlength{\rightskip}{0pt plus 5cm}def isEmpty ( {\em self})}\label{classsource__py_1_1_ring_buffer_1_1_ring_buffer_fb7514ca88d7e70499debb1302c4497d}


バッファが空であるか確認する 

バッファ空を確認する。

注)現在の実装では,現在のバッファ位置に格納されたデータが読み出されたか どうかを返す。( true:データ読み出し済,false:データ未読み出し)

\begin{Desc}
\item[引数:]
\begin{description}
\item[{\em self}]\end{description}
\end{Desc}
\begin{Desc}
\item[戻り値:]空確認結果 \end{Desc}


 RingBuffer.py の 214 行で定義されています。\index{source\_\-py::RingBuffer::RingBuffer@{source\_\-py::RingBuffer::RingBuffer}!isNew@{isNew}}
\index{isNew@{isNew}!source_py::RingBuffer::RingBuffer@{source\_\-py::RingBuffer::RingBuffer}}
\subsubsection{\setlength{\rightskip}{0pt plus 5cm}def isNew ( {\em self})}\label{classsource__py_1_1_ring_buffer_1_1_ring_buffer_a41773ac05767910a4cc7df88b7f814c}


最新データか確認する 

現在のバッファ位置に格納されているデータが最新データか確認する。

\begin{Desc}
\item[引数:]
\begin{description}
\item[{\em self}]\end{description}
\end{Desc}
\begin{Desc}
\item[戻り値:]最新データ確認結果 ( true:最新データ.データはまだ読み出されていない false:過去のデータ.データは既に読み出されている) \end{Desc}


 RingBuffer.py の 234 行で定義されています。\index{source\_\-py::RingBuffer::RingBuffer@{source\_\-py::RingBuffer::RingBuffer}!put@{put}}
\index{put@{put}!source_py::RingBuffer::RingBuffer@{source\_\-py::RingBuffer::RingBuffer}}
\subsubsection{\setlength{\rightskip}{0pt plus 5cm}def put ( {\em self},  {\em data})}\label{classsource__py_1_1_ring_buffer_1_1_ring_buffer_6b875be2179dd1eed37edd8d7fe079ff}


バッファにデータを格納する 

引数で与えられたデータをバッファに格納する。

注)現在の実装ではデータを格納すると同時に、データの読み出し位置を 格納したデータ位置に設定している。このため、常に直近に格納したデータを 取得する形となっている。

\begin{Desc}
\item[引数:]
\begin{description}
\item[{\em self}]\item[{\em data}]格納対象データ \end{description}
\end{Desc}


 RingBuffer.py の 257 行で定義されています。\index{source\_\-py::RingBuffer::RingBuffer@{source\_\-py::RingBuffer::RingBuffer}!get@{get}}
\index{get@{get}!source_py::RingBuffer::RingBuffer@{source\_\-py::RingBuffer::RingBuffer}}
\subsubsection{\setlength{\rightskip}{0pt plus 5cm}def get ( {\em self})}\label{classsource__py_1_1_ring_buffer_1_1_ring_buffer_444a1328efb32d5d9d2dcb2efe855d3b}


バッファからデータを取得する 

バッファに格納されたデータを取得する。

\begin{Desc}
\item[引数:]
\begin{description}
\item[{\em self}]\end{description}
\end{Desc}
\begin{Desc}
\item[戻り値:]取得データ \end{Desc}


 RingBuffer.py の 281 行で定義されています。\index{source\_\-py::RingBuffer::RingBuffer@{source\_\-py::RingBuffer::RingBuffer}!getRef@{getRef}}
\index{getRef@{getRef}!source_py::RingBuffer::RingBuffer@{source\_\-py::RingBuffer::RingBuffer}}
\subsubsection{\setlength{\rightskip}{0pt plus 5cm}def getRef ( {\em self})}\label{classsource__py_1_1_ring_buffer_1_1_ring_buffer_e1daf7a4fa2db844dc84df9a9bbf113c}


次に書き込むバッファへの参照を取得する 

書き込みバッファへの参照を取得する。

\begin{Desc}
\item[戻り値:]次の書き込み対象バッファへの参照\end{Desc}
\begin{Desc}
\item[引数:]
\begin{description}
\item[{\em self}]\end{description}
\end{Desc}


 RingBuffer.py の 301 行で定義されています。